\documentclass[a4paper,10pt]{article} % type, taille police

\usepackage[utf8]{inputenc} % encodage
\usepackage[T1]{fontenc} % encodage
\usepackage[french]{babel} % gestion du français
\usepackage{amssymb} % symboles mathématiques
\usepackage{textcomp} % flèche,  intervalle
\usepackage{stmaryrd} % intervalle entiers

\usepackage[left=3cm,right=3cm,top=3cm,bottom=3cm]{geometry} % marges
\usepackage[hidelinks]{hyperref} % sommaire interactif dans un pdf
\usepackage[nottoc, notlof, notlot]{tocbibind} % affichage des références dans la table des matières (?)
\usepackage{float} % placement des figures
\usepackage[toc,page]{appendix} % ajout d'annexes
\usepackage{amsthm} % format des déf, prop...
\usepackage{amsmath} % matrices, ...
\usepackage{multirow} % fusionner cellules verticalement

\usepackage{tikz} % affichage de schémas
\usepackage{graphicx} % affichage d'images
\usepackage{url} % inclure des urls
\usepackage{bbold} % fonction caractéristique 1

\usepackage{bussproofs} % arbre de preuves

\definecolor{bgreen}{rgb}{0.30,0.70,0}

\theoremstyle{definition} % pas d'italique pour le format des déf, prop...
\newtheorem{thm}{Théorème} % \begin{thm} \end{thm}
\newtheorem{cor}[thm]{Corollaire}
\newtheorem{defi}[thm]{Définition}
\newtheorem{ex}[thm]{Exemple}
\newtheorem{lem}[thm]{Lemme}
\newtheorem{rem}[thm]{Remarque}
\newtheorem{conj}[thm]{Conjecture}

\newcommand{\N}{\mathbb{N}}
\newcommand{\Z}{\mathbb{Z}}

\newcommand{\nr}{\neg_R}
\newcommand{\Nr}{^{\neg_R}}
\newcommand{\na}{\neg_A}
\newcommand{\Na}{^{\neg_A}}
\newcommand{\ra}{\Rightarrow}
\newcommand{\pr}{\vdash}
\newcommand{\nj}{_{NJ}}
\newcommand{\nk}{_{NK}}
\newcommand{\pa}{_{PA}}
\newcommand{\ha}{_{HA}}

\newcommand{\sN}{\text{set }\N_{\infty}}
\newcommand{\pS}{\text{proj1\_sig }}

%#############################################################################################################%
%#############################################################################################################%
%#############################################################################################################%

\title{DM3 - Preuves sur ordinateur \\ Omniscience}
\author{ {\Large Yassine \textsc{Hamoudi}}}
\date{25 novembre 2014}

\begin{document}

\maketitle

%#############################################################################################################%
%#############################################################################################################%
%#############################################################################################################%

\section{Remarques}

Les questions suivantes n'ont pas été résolues complètement, le code Coq correspondant comporte la tactique \texttt{admit} :
\begin{itemize}
 \item Question 4 : une preuve par induction a été essayée mais seule l'étape d'initialisation est démontrée.
 \item Question 15
 \item Question 17
\end{itemize}


%#############################################################################################################%


\section{Question 10}

On utilise le principe du tiers exclu.

D'après ce principe, on démontre que pour tout élément $x \in \sN$ :

\[ \exists k, \pS x \ k = false \vee \forall k, \pS x \ k = true\]

Considérons un élément $x \in \sN$.

Supposons qu'il existe $k \in \N$ tel que $\pS x \ k = false$. Alors : $\text{min } \pS x \ k = \text{false}$. Donc, d'après la question 6, il existe $p \in \N$ tel que $x = \text{of\_nat } p$.

A l'inverse, si $\forall k, \pS x \ k = true$ alors $x = \omega$.

Ceci démontre que $\N_{\infty} = \text{of\_nat}(\N) \cup \omega$ en logique classique.

%#############################################################################################################%

\section{Question 12}

On considère une énumération sur $\N$ des machines de Turing (chaque entier $n$ représente une machine de Turing notée $M_n$).

Pour tout $n \in \N$, on note $p_n : nat \rightarrow bool$ la fonction constamment vraie si $M_n$ s'arrête sur chacune de ses entrées, ou constamment fausse sinon. Le problème consistant à savoir si $p_n(m)$ est vraie ou faux (quelque soit $m$) est indécidable (il s'agit du problème de l'arrêt).

Supposons qu'il soit possible de prouver que $\N$ est omniscient dans une logique satisfaisant le théorème de disjonction. Alors, par définition de l'omniscience, on prouve que pour tout $n \in \N$ : $\exists m, p_n(m) = faux \vee \forall m, p_n(m) = vraie$. Cela implique, d'après le théorème de disjonction, que pour tout $n \in \N$ on ait une preuve constructive de $\exists m, p_n(m) = faux$ ou de $\forall m, p_n(m) = vraie$. Autrement dit, pour tout $n \in \N$ on parvient à décider si $M_n$ s'arrête sur chacune de ses entrées ou non, ce qui est impossible.

Il n'est donc pas possible de prouver que $\N$ est omniscient dans une logique satisfaisant le théorème de disjonction.

%#############################################################################################################%
%#############################################################################################################%
%#############################################################################################################%
\end{document}
